%iffalse
\let\negmedspace\undefined
\let\negthickspace\undefined
\documentclass[journal,12pt,twocolumn]{IEEEtran}
\usepackage{cite}
\usepackage{amsmath,amssymb,amsfonts,amsthm}
\usepackage{algorithmic}
\usepackage{graphicx}
\usepackage{textcomp}
\usepackage{xcolor}
\usepackage{txfonts}
\usepackage{listings}
\usepackage{enumitem}
\usepackage{mathtools}
\usepackage{gensymb}
\usepackage{comment}
\usepackage[breaklinks=true]{hyperref}
\usepackage{tkz-euclide} 
\usepackage{listings}
\usepackage{gvv}                                        
%\def\inputGnumericTable{}                                 
\usepackage[latin1]{inputenc}                                
\usepackage{color}                                            
\usepackage{array}                                            
\usepackage{longtable}                                       
\usepackage{calc}                                             
\usepackage{multirow}                                         
\usepackage{hhline}                                           
\usepackage{ifthen}                                           
\usepackage{lscape}
\usepackage{tabularx}
\usepackage{array}
\usepackage{float}


\newtheorem{theorem}{Theorem}[section]
\newtheorem{problem}{Problem}
\newtheorem{proposition}{Proposition}[section]
\newtheorem{lemma}{Lemma}[section]
\newtheorem{corollary}[theorem]{Corollary}
\newtheorem{example}{Example}[section]
\newtheorem{definition}[problem]{Definition}
\newcommand{\BEQA}{\begin{eqnarray}}
\newcommand{\EEQA}{\end{eqnarray}}
\newcommand{\define}{\stackrel{\triangle}{=}}
\theoremstyle{remark}
\newtheorem{rem}{Remark}
% Marks the beginning of the document
\begin{document}
\bibliographystyle{IEEEtran}
\vspace{3cm}

\title{Applications of Derivatives}
\author{ee24btech11066 - Akhilesh}
\maketitle
\newpage
\bigskip

\renewcommand{\thefigure}{\theenumi}
\renewcommand{\thetable}{\theenumi}
\begin{enumerate}
\item[25.] For $x \in \brak{0,\frac{5\pi}{2}}$, define $f(x)=\int\limits_0^x\sqrt{t}\sin t$ dt.Then $f$ has \hfill{[2011]}\\

   {(a)}  local minimum at $\pi$ and 2$\pi$
   
   {(b)}  local minimum at $\pi$ and local maximum at 2$\pi$

   {(c)}  local maximum at $\pi$ and local minimum at 2$\pi$

   {(d)}  local maximum at $\pi$ and 2$\pi$\\
    
\item[26.] A spherical balloon is filled with 4500$\pi$ cubic meters of helium gas. If a leak in the balloon causes the gas to escape at the rate of 72$\pi$ cubic meters per minute,then the rate{(in meters per minute)} at which the radius of the balloon decreases 49 minutes after the leakage began is $:$ \hfill{[2012]}\\
{(a)} $\frac{9}{7}$ \hspace{0.5cm} {(b)}  $\frac{7}{9}$ \hspace{0.5cm} {(c)}  $\frac{2}{9}$\hspace{0.5cm} {(d)} $\frac{9}{2}$ \\  

\item[27.] Let $a,b\in R$ be such that the function $f$ given by $f(x)=ln|x|+bx^{2}+ax,x=$!$0$ has extreme values at $x=-1$ and $x=2$\\
Statement-1 $: f$ has local maximum at $x$=-1 and at $x$=2.\\
Statement-2 $: a=\frac{1}{2}$ and $b=\frac{-1}{4}$ \hfill{[2012]}\\
{(a)} Statement-1 is false,Statement-2 is true.

{(b)} Statement-1 is true,Statement-2 is true;statement-2 is a correct explanation for Statement-1.

{(c)} Statement-1 is true,statement-2 is true;statement-2 is not a correct explanation for Statement-1.

{(d)} Statement-1 is true,statement-2 is false.\\

\item[28.] A line is drawn through the point(1,2) to meet the coordinate axes at $P$ and $Q$ such that it forms a triangle $OPQ,$where O is the origin.If the area of the triangle $OPQ$ is least,then the slope of the line $PQ$ is$:$  \hfill{[2012]}\\
{(a)}$\frac{-1}{4}$ \hspace{0.5cm}{(b)}-4\hspace{0.5cm}{(c)}-2\hspace{0.5cm}{(d)}$\frac{-1}{2}$\\
\item[29.] The intercepts on $x-$axis made by tangents to the curve, $y=\int\limits_0^x|t|dt,x\in R$,which are parallel to the line $y=2x,$ are equal to $:$ {[JEE M 2013]}\\
{(a)}$\pm1$\hspace{0.5cm}{(b)}$\pm2$\hspace{0.5cm}{(c)}$\pm3$\hspace{0.5cm}{(d)}$\pm4$\\

\item[30.] If $f and g$ are differentiable functions in {[0,1]} satisfying $f(0)=2=g(1),g(0) and f(1)=6$,then for some c$\in {[0,1]}$\hfill[JEE M 2014]
{(a)}$f'(c)=g'(c)$\hspace{1cm} {(b)}$f'(c)=2g'(c)$\\
{(c)}$2f'(c)=g'(c)$\hspace{1cm}{(d)}$2f'(c)=3g'(c)$\\
\item[31.] Let f(x) be a polynomial of degree four having extreme values at x=1 and x=2.If $\lim\limits_{x\to 0}$[1+$\frac{f(x)}{x^{2}}$]=3,then f(2) is equal to $:$

    \hfill[JEE M 2015]\\
{(a)} 0\hspace{0.5cm} {(b)}4\hspace{0.5cm} {(c)}-8\hspace{0.5cm} {(d)}-4 \\
\item[32.] Consider $:$\\
     f(x)=$\tan^{-1}\brak{ \sqrt{\frac{1+\sin{x}}{1-\sin{x}} }}$,$x \in \brak{0,\frac{\pi}{2}}$.
A normal to y=f(x) at x=$\frac{p}{6}$ also passes through the point$:$

\hfill[JEE M 2016]\\
{(a)}($\frac{\pi}{6},0$)\hspace{0.5cm} {(b)}($\frac{\pi}{4},0$)\hspace{0.5cm}{(c)}(0,0)\hspace{0.5cm}{(d)}($0,\frac{2\pi}{3}$)\\
\item[33.] A wire of length 2 units is cut into two parts which are bent respectively to form a square of side=x units and a circle of radius=r units.If the sum of the areas of the square and the circle so formed is minimum,then$:$

\hfill[JEE M 2016]\\
{(a)}x=2r\hspace{2cm}{(b)}2x=r   \\
{(c)}2x=$(\pi+4)r$\hspace{1cm}{(d)}(4$-\pi)x=\pi$r\\
\item[34.]The function f$: R\rightarrow{[\frac{-1}{2},\frac{1}{2}]}$ defined as f(x)=$\frac{x}{1+x^2}$,is $:$

\hfill[JEE M 2016]\\
{(a)} neither injective nor surjective\\
{(b)} invertible\\
{(c)} injective but not surjective\\
{(d)} surjective but not injective\\
\item[35.] The normal to the curve y(x-2)(x-3)=x+6 at the point where the curve intersects the $y$-axis passes through the point$:$\hfill[JEE M 2017]

{(a)}($\frac{1}{2},\frac{1}{3}$)\hspace{2cm}{(b)}($\frac{-1}{2},\frac{-1}{2}$)\\

{(c)}($\frac{1}{2},\frac{1}{2}$)\hspace{2cm}{(d)}($\frac{1}{2},\frac{-1}{3}$)\\
\item[36.] Twenty meters of wire is available for fencing off a flower-bed in the form of a circular sector.Then the maximum area(in sq.m) of the flower-bed, is$:$\hfill[JEE M 2017]\\
{(a)}30  \hspace{2cm}                    {(b)}12.5\\
{(c)}10  \hspace{2cm}                    {(d)}25\\
\item[37.]The eccentricity of an ellipse whose centre is at the origin is $\frac{1}{2}$.If one of its directices is x=-4,then the equation of the normal to it at (1,$\frac{3}{2}$) is $:$\hfill[JEE M 2017]\\
{(a)}x+2y=4\hspace{2cm}{(b)}2y-x=2\\
{(c)}4x-2y=1\hspace{2cm}{(d)}4x+2y=7\\
\item[38.] Let $f(x)=x^{2}+\frac{1}{x^2}$ and g(x)=x-$\frac{1}{x}$,$x \in R-{\{-1,0,1}\}$.If h(x)=$\frac{f(x)}{g(x)}$,then the local minimum value of h(x) is$:$\hfill[JEE M 2018]\\
{(a)}-3\hspace{0.5cm}{(b)}-2$\sqrt{2}$\hspace{0.5cm}{(c)}2$\sqrt{2}$\hspace{0.5cm}{(d)}3\\
\item[39.] If the curves $y^2=6x,9x^2+by^2=16$ intersect each other at right angles,then the value of b is$:$\hfill[JEE M 2018]\\
{(a)}$\frac{7}{2}$\hspace{1cm}{(b)}4\hspace{1cm}{(c)}$\frac{9}{2}$\hspace{1cm}{(d)}6

\end{enumerate}\

\end{document}

