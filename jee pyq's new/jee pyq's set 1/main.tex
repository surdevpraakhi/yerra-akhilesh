\let\negmedspace\undefined
\let\negthickspace\undefined
\documentclass[journal]{IEEEtran}
\usepackage[a5paper, margin=10mm, onecolumn]{geometry}
%\usepackage{lmodern} % Ensure lmodern is loaded for pdflatex
\usepackage{tfrupee} % Include tfrupee package

\setlength{\headheight}{1cm} % Set the height of the header box
\setlength{\headsep}{0mm}     % Set the distance between the header box and the top of the text

\usepackage{gvv-book}
\usepackage{gvv}
\usepackage{cite}
\usepackage{amsmath,amssymb,amsfonts,amsthm}
\usepackage{algorithmic}
\usepackage{graphicx}
\usepackage{textcomp}
\usepackage{xcolor}
\usepackage{txfonts}
\usepackage{listings}
\usepackage{enumitem}
\usepackage{mathtools}
\usepackage{gensymb}
\usepackage{comment}
\usepackage[breaklinks=true]{hyperref}
\usepackage{tkz-euclide} 
\usepackage{listings}
% \usepackage{gvv}                                        
\def\inputGnumericTable{}                                 
\usepackage[latin1]{inputenc}                                
\usepackage{color}                                            
\usepackage{array}                                            
\usepackage{longtable}                                       
\usepackage{calc}                                             
\usepackage{multirow}                                         
\usepackage{hhline}                                           
\usepackage{ifthen}                                           
\usepackage{lscape}
\begin{document}

\bibliographystyle{IEEEtran}
\vspace{3cm}

\title{september 4 shift2}
\author{EE24BTECH11066 - YERRA AKHILESH
}
% \maketitle
% \newpage
% \bigskip
{\let\newpage\relax\maketitle}

\renewcommand{\thefigure}{\theenumi}
\renewcommand{\thetable}{\theenumi}
\setlength{\intextsep}{10pt} % Space between text and floats


\numberwithin{equation}{enumi}
\numberwithin{figure}{enumi}
\renewcommand{\thetable}{\theenumi}
\begin{enumerate}[start=16]
\item The angle of elevation of a cloud $\vec{C}$ from a point $\vec{P}$, $200$m above a still lake is $30^{\circ}$. If the angle of depression of the image of $\vec{C}$ in the lake from the point $\vec{P}$ is $60^{\circ}$, then $PC$ \brak{\text{in m}} \hfill{[4th September shift 2,2020]}
\begin{enumerate}
\begin{multicols}{4}
\item 200$\sqrt{3}$
\item 400$\sqrt{3}$
\item 400
\item 100
\end{multicols}
\end{enumerate}

\item Let $\bigcup_{i=1}^{50} X_i = \bigcup_{i=1}^{n} Y_i = T$, where each $X_i$ contains 10 elements and each $Y_i$ contains 5 elements. If each element of the set $T$ is an element of exactly 20 of the sets $X_i$'s and exactly 6 of the sets $Y_i$'s, then  n is equal to : \hfill{[4th September shift 2,2020]}


\begin{enumerate}
\begin{multicols}{4}
\item 15
\item 30
\item 50
\item 45
\end{multicols}
\end{enumerate}

\item Let $x=4$ be a directrix to an ellipse whose centre is at the origin and its eccentricity is $\frac{1}{2}$. If $\vec{P}\brak{1,\beta},\beta>0$ is a point on this ellipse,then the equation of the normal to it at $\vec{P}$ is : \hfill{[4th September shift 2,2020]}
\begin{enumerate}
\begin{multicols}{4}
\item $8x-2y=5$
\item $4x-2y=1$
\item $7x-4y=1$
\item $4x-3y=2$
\end{multicols}
\end{enumerate}

\item Let $a_1, a_2, \ldots, a_n$ be a given A.P. whose common difference is an integer and $S_n = a_1 + a_2 + \ldots + a_n$. If $a_1 = 1$, $a_n = 300$ and $15 \leq n \leq 50$, then the ordered pair  $\brak{S_{n-4}, a_{n-4}}$ is equal to: \hfill{[4th September shift 2,2020]}

\begin{enumerate}
\begin{multicols}{4}
\item $\myvec{2480,248}$
\item $\myvec{2480,249}$
\item $\myvec{2490,249}$
\item $\myvec{2490,248}$
\end{multicols}
\end{enumerate}

\item The circle passing through the intersection of the circles, $x^2+y^2-6x=0$ and $x^2+y^2-4y=0$, having its centre on the line, $2x-3y+12=0$, also passes through the point: \hfill{[4th September shift 2,2020]}
\begin{enumerate}
\begin{multicols}{4}
\item $\myvec{-1,3}$
\item $\myvec{1,-3}$
\item $\myvec{-3,6}$
\item $\myvec{-3,1}$
\end{multicols}
\end{enumerate}

\item Let $\{x\}$ and ${[x]}$ denote the fractional part of $x$ and the greatest integer $\leq x$ respectively of a real number $x$. If $\int\limits_0^n\{x\} \, \text{dx},\int\limits_0^n{[x]} \,\text{dx}$ and $10\brak{n^2-n},\brak{n \in N,n>1}$ are three consecutive terms of a G.P., then $n$ is equal to \underline{\hspace{1cm}} \hfill{[4th September shift 2,2020]}\\ 

\item A test consists of $6$ multiple choice questions, each having $4$ alternative answers of which only one is correct. The number of ways, in which a candidate answers all six questions such that exactly four of the answers are correct, is \underline{\hspace{1cm}} \hfill{[4th September shift 2,2020]}\\

\item If $\bar{a} = 2\hat{i} + \hat{j} + 2\hat{k}$, then the value of $\abs{\hat{i} \times \brak{\bar{a} \times \hat{i}}}^2 + \abs{\hat{j} \times \brak{\bar{a} \times \hat{j}}}^2 + \abs{\hat{k} \times \brak{\bar{a} \times \hat{k}}}^2$  is equal to \underline{\hspace{1cm}} \hfill{[4th September shift 2,2020]}\\

\item Let $PQ$ be a diameter of the circle $x^2+y^2=9$. If $\alpha$ and $\beta$ are the lengths of the perpendiculars from $\vec{P}$ and $\vec{Q}$ on the straight line, $x+y=2$ respectively, then the maximum value of $\alpha\beta$ is \underline{\hspace{1cm}} \hfill{[4th September shift 2,2020]}\\

\item If the variance of the following frequency distribution:

\begin{center}
\begin{tabular}{|c|c|c|c|}
\hline
\textbf{Class}     & 10-20 & 20-30 & 30-40 \\
\hline
\textbf{Frequency} & 2     & x     & 2     \\
\hline
\end{tabular}
\end{center}

is $50$, then  x is equal to \underline{\hspace{1cm}} \hfill{[4th September shift 2,2020]}



\end{enumerate}

\end{document}
