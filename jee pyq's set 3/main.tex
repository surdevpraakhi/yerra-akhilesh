\let\negmedspace\undefined
\let\negthickspace\undefined
\documentclass[journal]{IEEEtran}
\usepackage[a5paper, margin=10mm, onecolumn]{geometry}
\usepackage{lmodern} % Ensure lmodern is loaded for pdflatex
\usepackage{tfrupee} % Include tfrupee package

\setlength{\headheight}{1cm} % Set the height of the header box
\setlength{\headsep}{0mm}     % Set the distance between the header box and the top of the text

\usepackage{gvv-book}
\usepackage{gvv}
\usepackage{cite}
\usepackage{amsmath,amssymb,amsfonts,amsthm}
\usepackage{algorithmic}
\usepackage{graphicx}
\usepackage{textcomp}
\usepackage{xcolor}
\usepackage{txfonts}
\usepackage{listings}
\usepackage{enumitem}
\usepackage{mathtools}
\usepackage{gensymb}
\usepackage{comment}
\usepackage[breaklinks=true]{hyperref}
\usepackage{tkz-euclide} 
\usepackage{listings}
\usepackage{gvv}                                        
\def\inputGnumericTable{}                                 
\usepackage[latin1]{inputenc}                                
\usepackage{color}                                            
\usepackage{array}                                            
\usepackage{longtable}                                       
\usepackage{calc}                                             
\usepackage{multirow}                                         
\usepackage{hhline}                                           
\usepackage{ifthen}                                           
\usepackage{lscape}
\begin{document}

\bibliographystyle{IEEEtran}
\vspace{3cm}

\title{2023 12th april shift2}
\author{EE24BTECH11066 - YERRA AKHILESH}
% \maketitle
% \newpage
% \bigskip
{\let\newpage\relax\maketitle}
\begin{enumerate}[start=16]
\item If the circles $x^2+y^2-2x-4y+4=0$ and $x^2+y^2-6x-10y+20+2\sqrt{13}=0$ touch each other at the point $\brak{a,b}$, then $\brak{3a-2b}^2$ is equal to :
\begin{enumerate}
    \item 1
    \item 4
    \item 9
    \item 13
\end{enumerate}

\item If the angle between the line $l:\frac{x-1}{2}=\frac{y+1}{1}=\frac{z-2}{2}$ and the plane $P:\lambda x+4y-7=0, \lambda\neq0$, is $\cosec^{-1}\brak{\frac{3}{2}}$, then the sum of co-ordinates of the point where line $l$ crosses the plane $P$  is :
\begin{enumerate}
    \item -33
    \item -2
    \item 3
    \item 6
\end{enumerate}

\item Let three distinct normal be drawn to the parabola $y^2+4y-6x-8=0$ from a point \brak{a,b} on the axis of the parabola. Then :
\begin{enumerate}
    \item $a \in \brak{1,\infty}$ and $b=-2$
    \item $a \in \brak{0,\infty}$ and $b=-2$
    \item $a \in \brak{1,\infty}$ and $b=2$
    \item $a \in \brak{2,\infty}$ and $b=2$
\end{enumerate}

\item Given ${}^{n}C_{r} = \frac{n!}{r!(n-r)!}, 0\leq r \leq n$ and $n$ is a non-negative integer. Then a possible value of k for which the equality\\
${}^{50}C_{k-1}+\sum_{r=1}^{50}{}^{100-r}C_{k-2}={}^{100}C_{49}$ holds, is :
\begin{enumerate}
    \item 40
    \item 49
    \item 50
    \item 25
\end{enumerate}

\item If an unbiased die, marked with $-3,-2,-1,0,1,2$ on its faces, is thrown four times, then the probability of getting $-1$ as the sum of outputs is :
\begin{enumerate}
    \item $\frac{7}{81}$
    \item $\frac{35}{324}$
    \item $\frac{8}{81}$
    \item $\frac{26}{243}$
\end{enumerate}

\item $\left( \sin 5^\circ \sin 55^\circ \sin 65^\circ \sin 75^\circ \right)^{-1}$ is equal to \underline{\hspace{1cm}}\\

\item If the shortest distance between the lines \\
$3x+2y-4z-5=0=5x-7y-17z+2$ and $\frac{x-2}{3}=\frac{y-1}{-5}=\frac{z+1}{2}$ is $\frac{10}{\sqrt{k}}$, then $k$ is equal to \underline{\hspace{1cm}}\\

\item For $p \in \mathbb N$, if the angle between pair of tangents drawn to the ellipse $3x^{2}+2y^{2}=5$ from the point \brak{1,p} is $\tan^{-1}\brak{\frac{12}{\sqrt{5}}}$, then the distance of the vertex of the parabola $y=x^2-px+p+1$ from the point \brak{-7,8} is equal to \underline{\hspace{1cm}}\\

\item Let $P$ be a polygon with $n$ vertices such that the line segment joining any two points of $P$ remains entirely in $P$. If the number of diagonals of $P$ is $n+25$, then $n$ is equal to \underline{\hspace{1cm}}\\

\item Let $f(x)$ be a polynomial of degree 5 such that $\lim\limits_{x\to 0}\frac{f(x)}{x^2}=1, f\brak{-1}=-1, f(x)-14x$ has an extrema at $x=1$ and $f(x)-10x$ has an extrema at $x=-1$. Then $f\brak{2}$ is equal to \underline{\hspace{1cm}}\\

\item The number of $7$ digits integers formed by using the digits $2,3,4,5$ only and having the sum of digits equal to $18$ is \underline{\hspace{1cm}}\\

\item The remainder when $\brak{556}^{40}$ is divided by $7$ is \underline{\hspace{1cm}}\\

\item Let ${[t]}$ denote the greatest integer less than or equal to $t$.Then the value of $10-10\int\limits_{-2}^{2}{[x+x^3]}\text{dx}$ is \underline{\hspace{1cm}}\\

\item Let $A=\frac{1}{\sqrt{2}}\myvec{1&1\\-1&1}$ and $B=\myvec{1&0\\0&-1}$. If $k, l \in \mathbb N$ be such that $A^{k}B^{l}=I$, then the minimum value of $k+l$ is \underline{\hspace{1cm}}\\

\item Let $A_1=\myvec{1}, A_2=\myvec{2&3\\4&5},A_3=\myvec{6&7&8\\9&10&11\\12&13&14},\cdots$ Then the sum of the diagonal elements of $A_{20}$ is \underline{\hspace{1cm}}\\

\end{enumerate}
\end{document}

