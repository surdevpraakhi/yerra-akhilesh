\let\negmedspace\undefined
\let\negthickspace\undefined
\documentclass[journal]{IEEEtran}
\usepackage[a5paper, margin=10mm, onecolumn]{geometry}
%\usepackage{lmodern} % Ensure lmodern is loaded for pdflatex
\usepackage{tfrupee} % Include tfrupee package

\setlength{\headheight}{1cm} % Set the height of the header box
\setlength{\headsep}{0mm}     % Set the distance between the header box and the top of the text

\usepackage{gvv-book}
\usepackage{gvv}
\usepackage{cite}
\usepackage{amsmath,amssymb,amsfonts,amsthm}
\usepackage{algorithmic}
\usepackage{graphicx}
\usepackage{textcomp}
\usepackage{xcolor}
\usepackage{txfonts}
\usepackage{listings}
\usepackage{enumitem}
\usepackage{mathtools}
\usepackage{gensymb}
\usepackage{comment}
\usepackage[breaklinks=true]{hyperref}
\usepackage{tkz-euclide} 
\usepackage{listings}
% \usepackage{gvv}                                        
\def\inputGnumericTable{}                                 
\usepackage[latin1]{inputenc}                                
\usepackage{color}                                            
\usepackage{array}                                            
\usepackage{longtable}                                       
\usepackage{calc}                                             
\usepackage{multirow}                                         
\usepackage{hhline}                                           
\usepackage{ifthen}                                           
\usepackage{lscape}
\begin{document}

\bibliographystyle{IEEEtran}
\vspace{3cm}

\title{9-9.2-20}
\author{EE24BTECH11066 - YERRA AKHILESH
}
% \maketitle
% \newpage
% \bigskip
{\let\newpage\relax\maketitle}

\renewcommand{\thefigure}{\theenumi}
\renewcommand{\thetable}{\theenumi}
\setlength{\intextsep}{10pt} % Space between text and floats


\numberwithin{equation}{enumi}
\numberwithin{figure}{enumi}
\renewcommand{\thetable}{\theenumi}
\textbf{Question}:\\
Find the area of the region bounded by the curves $y = x^2+2$, $y = x$, $x = 0$ and $x = 3$.
\\
\textbf{solution: }
\begin{table}[h!]
    \centering
    \begin{tabular}[12pt]{ |c| c|}
    \hline
    \textbf{point} & \textbf{Description}\\ 
    \hline
    $\vec{P}$ & \brak{5,-3} \\
    \hline
    $\vec{Q}$ & \brak{0,1} \\
    \hline
    $\vec{R}$ & $\brak{x,6}$ \\
    \hline
    \end{tabular}
\end{table}
The parameters of the conic are\\
\begin{table}[h!]
    \centering
    \begin{tabular}[12pt]{ |c| c|}
    \hline
    \textbf{Conic}& \textbf{Parameters}\\ 
    \hline
     $V$& $\myvec{0 & 0 \\ 0 & 1}$\\
    \hline 
     $u$& $\frac{-1}{2}\myvec{1\\0}$\\
    \hline
     $f$& $0$\\
     \hline
    \end{tabular}
    \label{Table2}
\end{table}

\begin{align}
    L : x_i=h+\kappa_i m 
\end{align}
Where,
\begin{align}
    \kappa_i=\frac{1}{m^\top Vm}(-m^\top(Vh+u) \pm \sqrt{[m^\top(Vh+u)]^2-g(h)(m^\top Vm)}
    \label{0.2}
\end{align}
For the curve $y = x$, the parameters are\\
\begin{align}    
    h_2=\myvec{0\\0} , m_2=\myvec{1\\1}
\end{align}
Substituting from the above in (\ref{0.2})\\
\begin{align}
    \kappa_i= 1,-1
\end{align}
yielding the points of intersection\\
\begin{align}
  a_0=\myvec{0\\0}, a_1=\myvec{3\\3}
\end{align}
Similarly, for the curve $y = x^2 + 2$ (\ref{0.2})\\
\begin{align}
   h_1=\myvec{0\\2} , m_1=\myvec{1\\x^2}
\end{align}
yielding\\
\begin{align}
    \kappa_i = 1, -1
\end{align}
from which, the points of intersection are\\
\begin{align}
      a_2=\myvec{0\\2}, a_3=\myvec{3\\11}
\end{align}
Thus, the area of the region between the curves  $y = x^2 + 2$ and $y = x$ from $x = 0$ to $x = 3$ is given by\\
\begin{align}
    \int_{0}^{3} \left| x - \brak{x^2 + 2} \right| \, dx
\end{align}
Evaluating the integral:
\begin{align}
    A = \int_{0}^{3} \left| x - \brak{x^2 + 2} \right| \, dx
\end{align}
Evaluating the expressions step by step:
\begin{align}
    A = \left[\frac{x^2}{2} - \frac{x^3}{3} - 2x\right]_{0}^{3}
\end{align}
Simplifying:
\begin{align}
    A = \left[\frac{9}{2} - \frac{27}{3} - 6\right] = \frac{9}{2} - 9 - 6 = \frac{21}{2}
\end{align}
Thus, the area enclosed is \(\frac{21}{2}\).


\begin{figure}[htp]
    \centering
    \includegraphics[width=10cm]{figs/Figure_1.png}
\end{figure}
\end{document}










